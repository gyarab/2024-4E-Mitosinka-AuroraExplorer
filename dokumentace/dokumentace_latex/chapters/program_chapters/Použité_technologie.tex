\section{Použité technologie}
\par Backend aplikace je postaven na výkonném JavaScriptovém runtime enviromentu Node.js a frameworku Express, který umožňuje snadnou tvorbu serverových aplikací. Node.js podporuje použití MVC architektury (Model-View-Controller), kterou jsem v mém projektu implementoval. Tato architektura přináší přehledné rozdělení logiky aplikace na tři hlavní vrstvy: modely, které reprezentují schéma jednotlivých modelů v databázi, kontrolery, které zpracovávají požadavky a upravují data v databázi, a šablony (view) ve formátu EJS,   které generují dynamický HTML obsah zobrazovaný uživatelům. Primárně používaný programovací jazyk je JavaScript, který se používá jak na straně serveru, tak na klientské části.
\par Jako databázový systém jsem zvolil MongoDB. MongoDB je NoSQL databáze, která ukládá data ve formátu JSON-like dokumentů (BSON). Na rozdíl od relačních databází (jako MySQL nebo PostgreSQL), MongoDB nevyužívá tabulky, ale místo toho ukládá data jako flexibilní dokumenty. Díky tomu je MongoDB ideální pro moji aplikaci, kde se struktura dat může často měnit.
\par Frontend aplikace je tvořen šablonovacím enginem EJS (Embedded JavaScript). EJS mi umožňuje jednoduše generovat obsah založený na aktuálních datech z databáze a přizpůsobit zobrazení konkrétním uživatelům.
\par Pro stylování aplikace jsem využil CSS framework Tailwind CSS, který mi umožňuje rychle a efektivně implementovat styly do HTML kódu, čímž jednoduše mohu zajistit vzhled a responzivitu jednotlivých stránek.

\subsection{Struktura aplikace}
\par Aplikace je dělena do jednotlivých modulů, které nesou odpovědnost za jednotlivé části aplikace. Seznam modulů je popsán na další straně.

\newpage

\begin{tabbing}
\hspace{5cm}\=\hspace{6.5cm}\=\kill
\textbf{Soubor} \> \textbf{Popis} \\[10pt]
\texttt{server.js} \> Spouští HTTPS server, definuje cesty, navazuje spojení s \\
                   \> databází, nastavuje middleware a spravuje statické soubory \\
                   \> a skripty. \\ [10pt]
\texttt{dbConfig.js} \> Configuruje připojení databáze. \\[10pt]
\texttt{postController.js} \> Ovladač pro operace s daty příspěvků. \\[10pt]
\texttt{userController.js} \> Ovladač spravující operace s uživatelskými daty. \\[10pt]
\texttt{attachUser.js} \> Authentikuje uživatele a připojení uživatele k objektu req.user. \\[10pt]
\texttt{logEvents.js} \> Zapisuje provedené požadavky na server. \\[10pt]
\texttt{upload.js} \> Nahrává soubory na server. \\[10pt]
\texttt{Post.js} \> Model příspěvku. \\[10pt]
\texttt{User.js} \> Model uživatele. \\[10pt]
\texttt{aurorexRoutes.js} \> Zpracovává cesty operací na sociální síti. \\[10pt]
\texttt{userRoutes.js} \> Zpracovává cesty uživatelských operací. \\[10pt]
\texttt{routes.js} \> Zpracovává hlavní cesty aplikace. \\[10pt]
\texttt{kpForecast.js} \> Zajišťuje získání dat o KP indexu a vytvoření grafů. \\[10pt]
\texttt{weatherData.js} \> Zajišťuje získání dat o počasí a jejich zobrazení. \\[10pt]
\texttt{kpNotificationService.js} \> Odesílá emaily při výskytu vysokého KP indexu. \\[10pt]
\texttt{notificationService.js} \> Odesílá emaily při zveřejnění příspěvku v lokaci uživatele. \\[10pt]
\end{tabbing}




