\section{Řešení problematiky}
\par Pozorování polární záře je krásný a vzrušující zážitek, ovšem spatřit onu podívanou na obloze může být velice obtížné. Najít vhodné místo, odkud je dobrý výhled a zároveň nízké světelné znečištění z měst, není jednoduché. Když vezmeme v potaz ještě oblačnost a výskyt vysokého KP index (KP index určuje intenzitu aktivity polární záře), stává se z toho poměrně komplikovaný úkol. Proto jsem vytvořil aplikaci Aurora Explorer, která umožňuje uživatelům vidět aktualní i budoucí data o intenzitě polární záře a oblačnosti. Spolu s aktuálními fotografiemi, pořízenými ostatními uživateli, a mapou výskytu těchto příspěvků se Aurora Explorer stává jedinečnou webovou aplikací pro pozorovatele polárních září.
\par Uživatelé bez přihlášení mají možnost vidět předpověď KP indexu a  procentuální oblačnost určenou podle jejich aktuální polohy. Také se mohou podívat do sekce pro tipy a rady, aby se mohli adekvátně na pozorování připravit. Po přihlášení se uživateli otevře možnost zhlédnout sekci Aurorex, která obsahuje fotografie pořízené a nahrané ostatními uživateli. Následně má možnost prohlédnout si příspěvky, které může okomentovat či označit, že se mu líbi. Kromě této sekce se zde nachází ještě živá mapa, na které je přehled všech lokací, které uživatelé označili ve svém příspěvku jako místo, odkud polární záři spatřili. Aby se nemohlo stát, že uživatel promešká příležitost vidět tento ohromující přírodní jev, může si ve svém profilu nastavit notifikace přes email v případě vysoké intenzity KP indexu. Uživatel má také možnost dostat notifikaci v případě, že v uživatelem zvolené libovolné lokaci je nahrán nový příspěvek.