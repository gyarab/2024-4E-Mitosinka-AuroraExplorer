\section{Hlavní části webové aplikace}

\subsection{Forecast}

\par Stránka /forecast je jednou z několika hlavních stránek, které má aplikace poskytuje. Zde má uživatel, i bez přihlášení, možnost vidět grafy s předpovědí KP indexu na aktuální i následující den spolu s dvěma mapami, které obsahují stávající a budoucí aktivitu polární záře na severní polokouli. Také je možnost prohlédnout si informace o počasí, jako jsou teplota, vlhkost vzduchu, větrnost a hlavně oblačnost, která je jedním z nejvíce ovlivňujících faktorů viditelnosti polární záře.
\par Data na počasí jsou čerpána z OpenWeatherMap pomocí jejich Weather API. Data o aktuálním počasí jsou zobrazena v elementu `\texttt{<div id="current-weather"\>>}` a předpověď v elementu `\texttt{<div id="forecast-weather"\>>}`. Aktualizaci dat v elementech zajišťuje script weatherData.js, který je inicializován ihned při načtení stránky. Pro správné fungování dat o počasí je potřeba povolit aplikaci informace o aktuální poloze, o které při načtení sama zažadá, jelikož data o počasí jsou hledána podle aktuální polohy uživatele. V případě nepovolení nebo nereagování na žádost data nebudou načtena.
\par Data pro grafy s hodnotami KP indexu jsou získávána z NOAA Space Weather Prediction Center. Obdržená data jsou následně zpracována ve scriptu kpForecast.js, ze kterého jsou poté načtena. Na samotné vytvoření grafu používám knihovnu Chart.js. Každý graf má vlastní element; graf pro dnešek je id="kpChart1" a pro zítřek se jedná o "kpChart2".

\begin{lstlisting}[caption = {Element s aktuální tabulkou KP indexu},label = {lst:stranka}]
<canvas id="kpChart1" class="w-full h-64"></canvas>
\end{lstlisting}

\par Fotografie s mapou výskytu polární záře na severní polokouli jsou také z NOAA Space Weather Prediction Center. Každá fotografie je uložena v elementu.

\begin{lstlisting}[caption = {Element s aktuální mapou polární záře},label = {lst:stranka}]
<img id="aurora-map-today" src="https://services.swpc.noaa.gov/images/animations/ovation/north/latest.jpg" class="absolute inset-0 w-full h-full object-cover">
\end{lstlisting}

Fotografie jsou automaticky aktualizovány pomocí funkce `\texttt{refreshImages()}`, která zajišťuje obnovení fotografií každých 10 minut. Obě mapy obsahují malou 'i' ikonku, která odkazuje na zdroj daného obrázku.

\subsection{Aurorex}

\par /Aurorex je další esenciální stránka pro můj program. Nejprve se uživatel musí přihlásit, aby bylo možné zpřístupnit tuto stránku. /Aurorex je samotnou sociální sítí, kde mohou uživatelé nahrávát a zobrazovat příspěvky, commentovat je, či likovat. Obsahuje také postupné, nekonečné načítání již nahraných příspěvků.
\par V případě, že se uživatel snaží zpřístupnit stránku bez přihlašení, bude přesměrován na /login, tedy stránku, která slouží k přihlášení. Po přihlášení se uživateli zobrazí stránka, v jejíž horní části je možnost Create Post, kterou je uživatel přesměrován na /aurorex/post, kde si může nadefinovat svůj příspěvek a zveřejnit ho na síti. Vedle této možnosti se nachazí filtr.
\par Při nahrávání příspěvku uživatel musí zvolit místo na mapě nebo napsat adresu ručně do vyznačeného pole a vybrat fotografii, kterou chce publikovat. Následně má možnost napsat popisek a poté fotografii nahrát na server. Po nahrátí jsou data uložena do databáze a samotná fotografie, kterou do databáze nelze nahrát, je uložena middlewarem upload.js do složky /public/uploads pomocí javascriptové knihovny \texttt{multer} a cesta k dané fotografii je uložena do databáze \cite{multer_file_upload} \cite{multer_file_upload_tutorial} \cite{multer_npm}. 
\par Filtr slouží k možnosti seřazení příspěvků dle tří parametrů; dle času od nejnovějších, podle počtu liků a podle počtu commentářů, z čehož defaultně je nastaven filtr newest, tedy od nejnovějšího příspěvku.
\par Níže se uživateli načítají příspěvky pomocí infinite scrollu. Tedy když se uživatel posune na konec stránky, spustí se funkce `\texttt{loadMorePosts()}`, která odešle požadavek na server s aktuální stránkou, maximálním počtem načtených příspěvků na jeden request, tedy 6 příspevků a aktuálně vybraným filtrem. Server vrátí další sadu příspěvků, které se následně zobrazí níže na stránce. Pokud již nejsou dostupné další příspěvky, načítání se zastaví. Při změně filtru se seznam příspěvků resetuje a načítá se znovu od začátku. Pro zabránění opakovaného načítání jsou zde implementovány 2 proměnné. Proměnná `\texttt{loading}` pomocí true nebo false určuje, jestli se příspěvky aktuálně načítají a proměnná `\texttt{hasMore}` kontroluje, zdali existují další příspěvky, které by bylo možné zobrazit.
\par Následně u každého příspěvku je potřeba přeměnit souřadnice uložené v databázi na uživatelsky čitelnou lokaci. To zajišťuje funkce `\texttt{geocoder()}`. Ta nejprve najde všechny elementy `\texttt{location-display}` a vezme zeměpisnou šířku a délku uloženou v jejich datech. Pomocí třídy `\texttt{Geocoder()}` z Google Maps JavaScript API knihovy mám možnost přes funkci `\texttt{geocode()}` konvertovat souřadnice do adres, které následně zobrazím v elementu `\texttt{location-text}` \cite{latlong_to_address} \cite{geoapify_geocoding}.

\subsection{Live-map}

\par Webová stránka /aurorex/live-map slouží k zobrazení mapy, na které jsou označené lokace aktuálních příspěvků nahraných uživateli. Stejně jako u /aurorex se uživatel musí nejprve přihlásit, aby mu byl umožněn přístup na stránku. Po příhlášení se uživateli zobrazí mapa s příspěvky nahranými za posledních 24 hodin. Příspěvky jsou na mapě zobrazeny ve formě markeru, který se po kliknutí zvětší a zobrazí se uživateli pole s fotkou příspěvku, jménem uživatele, co příspěvek nahrál, datumem nahrání a popiskem, který je k příspěvku při jeho vytváření vložen.
\par Při načtení stránky se zavolá funkce `\texttt{initMap()}`, která pomocí třídy `\texttt{Map()}` z Google Maps JavaScript API knihovny  inicializuje mapu. Funkce následně zavolá další funkci `\texttt{loadPosts()}`, která odešle požadavek na server spolu se zvoleným časovým filtrem.

\begin{lstlisting}[caption = {Cesta pro vyžádání příspěvků pro live-map},label = {lst:stranka}]
const response = await fetch(`/aurorex/api/posts-by-time?range=${range}`);
\end{lstlisting}

Následně router v aurorexRoutes na odeslaný požadavek zareaguje zavoláním funkce ze souboru postController `\texttt{timeMapFilter}`. Ta spočítá čas `\texttt{startDate}`, od kterého se mají příspěvky načítat, pomocí zadaného parametru \$range a s využitím operátoru \$gte (greater than or equal) implementovaném v MongoDB najde veškeré příspěvky v kolekci posts, které mají čas vytvoření větší než spočítaný `\texttt{startDate}`. Následně vratí vytříděné příspěvky ve formě JSON a ty jsou zobrazeny na stránce ve formě markerů.
\par Google Maps JavaScript API knihovna poskytuje třídy `\texttt{Marker()}` a `\texttt{InfoWindow()}`. Třída `\texttt{Marker()}` mi umožňuje vytvořit samotný marker a nadefinovat si jeho vzhled a lokaci na mapě a třída `\texttt{InfoWindow()}` jednoduše vytvoří po kliknutí na marker malé okénko, jehož obsah si mohu customizovat a vložit do něj požadované informace o příspěvku \cite{google_maps_infowindow}.
\par Filtrovat příslevky podle času lze podle tří kritérií; příspěvky za posledních 24 hodin ('24 hours'), za poslední týden ('past week') a za poslední měsíc ('past month'). 

\subsection{Guide}

\par Stránka /guide je určena primárně jako informační stránká, kde se nachazí informace, které jsou esenciální či mohou být užitečné při výpravě za pozorováním polární záře. Obsahuje informace o nejlepších lokalitách, optimálním čase, tipech na fotografování a nezbytné přípravě pro úspěšné sledování aurory borealis. /Guide pomáhá uživatelům lépe se orientovat v podmínkách esenciálních k jejímu pozorování a poskytuje praktické rady, jak se na výpravu připravit a jak se vrátit s nejlepšími zaběry. 